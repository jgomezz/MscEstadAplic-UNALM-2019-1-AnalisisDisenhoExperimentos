\documentclass[]{article}
\usepackage{lmodern}
\usepackage{amssymb,amsmath}
\usepackage{ifxetex,ifluatex}
\usepackage{fixltx2e} % provides \textsubscript
\ifnum 0\ifxetex 1\fi\ifluatex 1\fi=0 % if pdftex
  \usepackage[T1]{fontenc}
  \usepackage[utf8]{inputenc}
\else % if luatex or xelatex
  \ifxetex
    \usepackage{mathspec}
  \else
    \usepackage{fontspec}
  \fi
  \defaultfontfeatures{Ligatures=TeX,Scale=MatchLowercase}
\fi
% use upquote if available, for straight quotes in verbatim environments
\IfFileExists{upquote.sty}{\usepackage{upquote}}{}
% use microtype if available
\IfFileExists{microtype.sty}{%
\usepackage{microtype}
\UseMicrotypeSet[protrusion]{basicmath} % disable protrusion for tt fonts
}{}
\usepackage[margin=1in]{geometry}
\usepackage{hyperref}
\hypersetup{unicode=true,
            pdftitle={Examen Final de Análisis y Diseños de Experimentos},
            pdfauthor={Jaime Gomez Marin},
            pdfborder={0 0 0},
            breaklinks=true}
\urlstyle{same}  % don't use monospace font for urls
\usepackage{color}
\usepackage{fancyvrb}
\newcommand{\VerbBar}{|}
\newcommand{\VERB}{\Verb[commandchars=\\\{\}]}
\DefineVerbatimEnvironment{Highlighting}{Verbatim}{commandchars=\\\{\}}
% Add ',fontsize=\small' for more characters per line
\usepackage{framed}
\definecolor{shadecolor}{RGB}{248,248,248}
\newenvironment{Shaded}{\begin{snugshade}}{\end{snugshade}}
\newcommand{\AlertTok}[1]{\textcolor[rgb]{0.94,0.16,0.16}{#1}}
\newcommand{\AnnotationTok}[1]{\textcolor[rgb]{0.56,0.35,0.01}{\textbf{\textit{#1}}}}
\newcommand{\AttributeTok}[1]{\textcolor[rgb]{0.77,0.63,0.00}{#1}}
\newcommand{\BaseNTok}[1]{\textcolor[rgb]{0.00,0.00,0.81}{#1}}
\newcommand{\BuiltInTok}[1]{#1}
\newcommand{\CharTok}[1]{\textcolor[rgb]{0.31,0.60,0.02}{#1}}
\newcommand{\CommentTok}[1]{\textcolor[rgb]{0.56,0.35,0.01}{\textit{#1}}}
\newcommand{\CommentVarTok}[1]{\textcolor[rgb]{0.56,0.35,0.01}{\textbf{\textit{#1}}}}
\newcommand{\ConstantTok}[1]{\textcolor[rgb]{0.00,0.00,0.00}{#1}}
\newcommand{\ControlFlowTok}[1]{\textcolor[rgb]{0.13,0.29,0.53}{\textbf{#1}}}
\newcommand{\DataTypeTok}[1]{\textcolor[rgb]{0.13,0.29,0.53}{#1}}
\newcommand{\DecValTok}[1]{\textcolor[rgb]{0.00,0.00,0.81}{#1}}
\newcommand{\DocumentationTok}[1]{\textcolor[rgb]{0.56,0.35,0.01}{\textbf{\textit{#1}}}}
\newcommand{\ErrorTok}[1]{\textcolor[rgb]{0.64,0.00,0.00}{\textbf{#1}}}
\newcommand{\ExtensionTok}[1]{#1}
\newcommand{\FloatTok}[1]{\textcolor[rgb]{0.00,0.00,0.81}{#1}}
\newcommand{\FunctionTok}[1]{\textcolor[rgb]{0.00,0.00,0.00}{#1}}
\newcommand{\ImportTok}[1]{#1}
\newcommand{\InformationTok}[1]{\textcolor[rgb]{0.56,0.35,0.01}{\textbf{\textit{#1}}}}
\newcommand{\KeywordTok}[1]{\textcolor[rgb]{0.13,0.29,0.53}{\textbf{#1}}}
\newcommand{\NormalTok}[1]{#1}
\newcommand{\OperatorTok}[1]{\textcolor[rgb]{0.81,0.36,0.00}{\textbf{#1}}}
\newcommand{\OtherTok}[1]{\textcolor[rgb]{0.56,0.35,0.01}{#1}}
\newcommand{\PreprocessorTok}[1]{\textcolor[rgb]{0.56,0.35,0.01}{\textit{#1}}}
\newcommand{\RegionMarkerTok}[1]{#1}
\newcommand{\SpecialCharTok}[1]{\textcolor[rgb]{0.00,0.00,0.00}{#1}}
\newcommand{\SpecialStringTok}[1]{\textcolor[rgb]{0.31,0.60,0.02}{#1}}
\newcommand{\StringTok}[1]{\textcolor[rgb]{0.31,0.60,0.02}{#1}}
\newcommand{\VariableTok}[1]{\textcolor[rgb]{0.00,0.00,0.00}{#1}}
\newcommand{\VerbatimStringTok}[1]{\textcolor[rgb]{0.31,0.60,0.02}{#1}}
\newcommand{\WarningTok}[1]{\textcolor[rgb]{0.56,0.35,0.01}{\textbf{\textit{#1}}}}
\usepackage{graphicx,grffile}
\makeatletter
\def\maxwidth{\ifdim\Gin@nat@width>\linewidth\linewidth\else\Gin@nat@width\fi}
\def\maxheight{\ifdim\Gin@nat@height>\textheight\textheight\else\Gin@nat@height\fi}
\makeatother
% Scale images if necessary, so that they will not overflow the page
% margins by default, and it is still possible to overwrite the defaults
% using explicit options in \includegraphics[width, height, ...]{}
\setkeys{Gin}{width=\maxwidth,height=\maxheight,keepaspectratio}
\IfFileExists{parskip.sty}{%
\usepackage{parskip}
}{% else
\setlength{\parindent}{0pt}
\setlength{\parskip}{6pt plus 2pt minus 1pt}
}
\setlength{\emergencystretch}{3em}  % prevent overfull lines
\providecommand{\tightlist}{%
  \setlength{\itemsep}{0pt}\setlength{\parskip}{0pt}}
\setcounter{secnumdepth}{0}
% Redefines (sub)paragraphs to behave more like sections
\ifx\paragraph\undefined\else
\let\oldparagraph\paragraph
\renewcommand{\paragraph}[1]{\oldparagraph{#1}\mbox{}}
\fi
\ifx\subparagraph\undefined\else
\let\oldsubparagraph\subparagraph
\renewcommand{\subparagraph}[1]{\oldsubparagraph{#1}\mbox{}}
\fi

%%% Use protect on footnotes to avoid problems with footnotes in titles
\let\rmarkdownfootnote\footnote%
\def\footnote{\protect\rmarkdownfootnote}

%%% Change title format to be more compact
\usepackage{titling}

% Create subtitle command for use in maketitle
\providecommand{\subtitle}[1]{
  \posttitle{
    \begin{center}\large#1\end{center}
    }
}

\setlength{\droptitle}{-2em}

  \title{Examen Final de Análisis y Diseños de Experimentos}
    \pretitle{\vspace{\droptitle}\centering\huge}
  \posttitle{\par}
    \author{Jaime Gomez Marin}
    \preauthor{\centering\large\emph}
  \postauthor{\par}
    \date{}
    \predate{}\postdate{}
  

\begin{document}
\maketitle

\hypertarget{en-un-articulo-de-solid-state-technology-diseno-ortogonal-para-optimizacion-de-procesos-y-su-aplicacion-en-el-grabado-quimico-con-plasma-se-describe-la-aplicacion-de-disenos-factoriales-en-el-desarrollo-de-un-proceso-de-grabado-quimico-con-nitruros-en-un-dispositivo-de-grabado-quimico-con-plasma-para-una-sola-oblea-el-proceso-usa-c2f6-como-gas-de-reaccion.-cuatro-factores-son-de-interes-el-entrehierro-anodo-catodo-a-la-presion-en-la-camara-del-reactor-b-el-flujo-del-gas-c2f6-c-y-la-potencia-aplicada-al-catodo-d.-la-respuesta-de-interes-es-la-rapidez-de-grabado-para-el-nitruro-de-silicio.-se-corre-una-sola-replica-de-un-diseno-24-los-datos-se-muestran-enseguida}{%
\paragraph{1.- En un artículo de Solid State Technology (``Diseño
ortogonal para optimización de procesos y su aplicación en el grabado
químico con plasma'') se describe la aplicación de diseños factoriales
en el desarrollo de un proceso de grabado químico con nitruros en un
dispositivo de grabado químico con plasma para una sola oblea El proceso
usa C2F6 como gas de reacción. Cuatro factores son de interés: el
entrehierro ánodo-cátodo (A), la presión en la cámara del reactor (B),
el flujo del gas C2F6 (C) y la potencia aplicada al cátodo (D). La
respuesta de interés es la rapidez de grabado para el nitruro de
silicio. Se corre una sola réplica de un diseño 24; los datos se
muestran
enseguida:}\label{en-un-articulo-de-solid-state-technology-diseno-ortogonal-para-optimizacion-de-procesos-y-su-aplicacion-en-el-grabado-quimico-con-plasma-se-describe-la-aplicacion-de-disenos-factoriales-en-el-desarrollo-de-un-proceso-de-grabado-quimico-con-nitruros-en-un-dispositivo-de-grabado-quimico-con-plasma-para-una-sola-oblea-el-proceso-usa-c2f6-como-gas-de-reaccion.-cuatro-factores-son-de-interes-el-entrehierro-anodo-catodo-a-la-presion-en-la-camara-del-reactor-b-el-flujo-del-gas-c2f6-c-y-la-potencia-aplicada-al-catodo-d.-la-respuesta-de-interes-es-la-rapidez-de-grabado-para-el-nitruro-de-silicio.-se-corre-una-sola-replica-de-un-diseno-24-los-datos-se-muestran-enseguida}}

\begin{verbatim}
       | (-) bajo   1 | (+) alto
\end{verbatim}

----------\textbar{}-------\textbar{}----------------- A (cm) \textbar{}
0.80 \textbar{} 1.20 B (mTorr) \textbar{} 450 \textbar{} 550 C (SCCM)
\textbar{} 125 \textbar{} 200 D (W) \textbar{} 275 \textbar{} 325

\begin{enumerate}
\def\labelenumi{(\arabic{enumi})}
\tightlist
\item
  550 a = 669 b = 604 ab = 650 c = 633 ac = 642 bc = 601 abc = 635 d =
  1037 ad = 749 bd = 1052 abd = 868 cd = 1075 acd = 860 bcd = 1063 abcd
  = 759
\end{enumerate}

y=c(550,669,604,650,633,642,601,635,1037,749,1052,868,1075,860,1063,759)

\hypertarget{a.estimar-los-efectos-de-los-factores-y-obtenga-una-grafica-de-probabilidad-normal-de-los-efectos-estimados-de-estos-factores.-que-efectos-parecen-ser-grande-6-puntos}{%
\paragraph{a).Estimar los efectos de los factores y obtenga una gráfica
de probabilidad normal de los efectos estimados de estos factores. ¿Que
efectos parecen ser grande? ( 6
puntos)}\label{a.estimar-los-efectos-de-los-factores-y-obtenga-una-grafica-de-probabilidad-normal-de-los-efectos-estimados-de-estos-factores.-que-efectos-parecen-ser-grande-6-puntos}}

\begin{Shaded}
\begin{Highlighting}[]
\CommentTok{#}
\NormalTok{y=}\KeywordTok{c}\NormalTok{(}\DecValTok{550}\NormalTok{,}\DecValTok{669}\NormalTok{,}\DecValTok{604}\NormalTok{,}\DecValTok{650}\NormalTok{,}\DecValTok{633}\NormalTok{,}\DecValTok{642}\NormalTok{,}\DecValTok{601}\NormalTok{,}\DecValTok{635}\NormalTok{,}\DecValTok{1037}\NormalTok{,}\DecValTok{749}\NormalTok{,}\DecValTok{1052}\NormalTok{,}\DecValTok{868}\NormalTok{,}\DecValTok{1075}\NormalTok{,}\DecValTok{860}\NormalTok{,}\DecValTok{1063}\NormalTok{,}\DecValTok{759}\NormalTok{)}

\CommentTok{#}
\NormalTok{A <-}\StringTok{ }\KeywordTok{rep}\NormalTok{(}\KeywordTok{c}\NormalTok{(}\KeywordTok{rep}\NormalTok{(}\OperatorTok{-}\DecValTok{1}\NormalTok{, }\DecValTok{1}\NormalTok{) , }\KeywordTok{rep}\NormalTok{(}\DecValTok{1}\NormalTok{, }\DecValTok{1}\NormalTok{)) , }\DecValTok{8}\NormalTok{)}
\NormalTok{B <-}\StringTok{ }\KeywordTok{rep}\NormalTok{(}\KeywordTok{c}\NormalTok{(}\KeywordTok{rep}\NormalTok{(}\OperatorTok{-}\DecValTok{1}\NormalTok{, }\DecValTok{2}\NormalTok{) , }\KeywordTok{rep}\NormalTok{(}\DecValTok{1}\NormalTok{, }\DecValTok{2}\NormalTok{)) , }\DecValTok{4}\NormalTok{)}
\NormalTok{C <-}\StringTok{ }\KeywordTok{rep}\NormalTok{(}\KeywordTok{c}\NormalTok{(}\KeywordTok{rep}\NormalTok{(}\OperatorTok{-}\DecValTok{1}\NormalTok{, }\DecValTok{4}\NormalTok{) , }\KeywordTok{rep}\NormalTok{(}\DecValTok{1}\NormalTok{, }\DecValTok{4}\NormalTok{)) , }\DecValTok{2}\NormalTok{)}
\NormalTok{D <-}\StringTok{ }\KeywordTok{rep}\NormalTok{(}\KeywordTok{c}\NormalTok{(}\KeywordTok{rep}\NormalTok{(}\OperatorTok{-}\DecValTok{1}\NormalTok{, }\DecValTok{8}\NormalTok{) , }\KeywordTok{rep}\NormalTok{(}\DecValTok{1}\NormalTok{, }\DecValTok{8}\NormalTok{)) , }\DecValTok{1}\NormalTok{)}

\CommentTok{#}
\KeywordTok{printmat}\NormalTok{(}\StringTok{"y "}\NormalTok{,y)}
\end{Highlighting}
\end{Shaded}

\begin{verbatim}
## [1] "y :  [1]  550  669  604  650  633  642  601  635 1037  749 1052  868 1075  860"
## [2] "y : [15] 1063  759"
\end{verbatim}

\begin{Shaded}
\begin{Highlighting}[]
\KeywordTok{printmat}\NormalTok{(}\StringTok{"A "}\NormalTok{,A)}
\end{Highlighting}
\end{Shaded}

\begin{verbatim}
## [1] "A :  [1] -1  1 -1  1 -1  1 -1  1 -1  1 -1  1 -1  1 -1  1"
\end{verbatim}

\begin{Shaded}
\begin{Highlighting}[]
\KeywordTok{printmat}\NormalTok{(}\StringTok{"B "}\NormalTok{,B)}
\end{Highlighting}
\end{Shaded}

\begin{verbatim}
## [1] "B :  [1] -1 -1  1  1 -1 -1  1  1 -1 -1  1  1 -1 -1  1  1"
\end{verbatim}

\begin{Shaded}
\begin{Highlighting}[]
\KeywordTok{printmat}\NormalTok{(}\StringTok{"C "}\NormalTok{,C)}
\end{Highlighting}
\end{Shaded}

\begin{verbatim}
## [1] "C :  [1] -1 -1 -1 -1  1  1  1  1 -1 -1 -1 -1  1  1  1  1"
\end{verbatim}

\begin{Shaded}
\begin{Highlighting}[]
\KeywordTok{printmat}\NormalTok{(}\StringTok{"D "}\NormalTok{,D)}
\end{Highlighting}
\end{Shaded}

\begin{verbatim}
## [1] "D :  [1] -1 -1 -1 -1 -1 -1 -1 -1  1  1  1  1  1  1  1  1"
\end{verbatim}

\begin{Shaded}
\begin{Highlighting}[]
\NormalTok{mod <-}\StringTok{ }\KeywordTok{lm}\NormalTok{(y}\OperatorTok{~}\NormalTok{A}\OperatorTok{*}\NormalTok{B}\OperatorTok{*}\NormalTok{C}\OperatorTok{*}\NormalTok{D)}
\NormalTok{Estimados <-}\StringTok{ }\DecValTok{2}\OperatorTok{*}\KeywordTok{coefficients}\NormalTok{(mod)[}\OperatorTok{-}\DecValTok{1}\NormalTok{]  }\CommentTok{# Se retira la columna del intercepto}
\CommentTok{# Estimados}
\KeywordTok{round}\NormalTok{(Estimados, }\DecValTok{0}\NormalTok{)}
\end{Highlighting}
\end{Shaded}

\begin{verbatim}
##       A       B       C       D     A:B     A:C     B:C     A:D     B:D 
##     -98       2      11     310      -4     -21     -40    -150       3 
##     C:D   A:B:C   A:B:D   A:C:D   B:C:D A:B:C:D 
##       2     -12       8       9     -22     -36
\end{verbatim}

Hay que buscar los estimados que mas aportan al modelo

\begin{Shaded}
\begin{Highlighting}[]
\KeywordTok{data.frame}\NormalTok{(}\KeywordTok{round}\NormalTok{(Estimados, }\DecValTok{0}\NormalTok{))}
\end{Highlighting}
\end{Shaded}

\begin{verbatim}
##         round.Estimados..0.
## A                       -98
## B                         2
## C                        11
## D                       310
## A:B                      -4
## A:C                     -21
## B:C                     -40
## A:D                    -150
## B:D                       3
## C:D                       2
## A:B:C                   -12
## A:B:D                     8
## A:C:D                     9
## B:C:D                   -22
## A:B:C:D                 -36
\end{verbatim}

Quienes mas aportan son: 1,4,6,7,8,14,15

\begin{Shaded}
\begin{Highlighting}[]
\NormalTok{qq <-}\StringTok{ }\KeywordTok{qqnorm}\NormalTok{(Estimados,}\DataTypeTok{type=}\StringTok{"n"}\NormalTok{)}
\NormalTok{Efectos <-}\StringTok{ }\KeywordTok{names}\NormalTok{(Estimados)}
\KeywordTok{text}\NormalTok{(qq}\OperatorTok{$}\NormalTok{x, qq}\OperatorTok{$}\NormalTok{y, }\DataTypeTok{labels =}\NormalTok{ Efectos)}
\NormalTok{Estimados1 <-}\StringTok{ }\NormalTok{Estimados[}\OperatorTok{-}\KeywordTok{c}\NormalTok{(}\DecValTok{1}\NormalTok{,}\DecValTok{4}\NormalTok{,}\DecValTok{6}\NormalTok{,}\DecValTok{7}\NormalTok{,}\DecValTok{8}\NormalTok{,}\DecValTok{14}\NormalTok{,}\DecValTok{15}\NormalTok{)] }\CommentTok{# SON LOS VALORES QUE MAS APORTAN AL MODELO}
\CommentTok{# Estimados1 <- Estimados}
\KeywordTok{qqline}\NormalTok{(Estimados1)}
\end{Highlighting}
\end{Shaded}

\includegraphics{efade2019i_files/figure-latex/unnamed-chunk-4-1.pdf} De
la grafica se puede apreciar que los siguiente efectos son los mas
grandes en orden de influencia:

\begin{itemize}
\tightlist
\item
  D : La potencia aplicada al cátodo
\item
  A*D : El entrehierro ánodo-cátodo y la potencia aplicada al cátodo
\item
  A : El entrehierro ánodo-cátodo
\item
  B*C : La presión en la cámara del reacto y el flujo del gas C2F6
\end{itemize}

Hay mas efectos como: A\emph{B}C + B\emph{C}D , pero los anteriores
tiene un mayor efecto.

\hypertarget{b.--realice-el-analisis-de-varianza-para-confirmar-los-resultados-obtenidos-en-a-obtenga-la-ecuacion-de-regresion-estimada-4-puntos}{%
\paragraph{b).- Realice el análisis de varianza para confirmar los
resultados obtenidos en (a), Obtenga la ecuación de regresión estimada
(4
puntos)}\label{b.--realice-el-analisis-de-varianza-para-confirmar-los-resultados-obtenidos-en-a-obtenga-la-ecuacion-de-regresion-estimada-4-puntos}}

\begin{Shaded}
\begin{Highlighting}[]
\CommentTok{# Con la información de los efectos detectados en el apartado anterior procedemos a  construimos nuestro modelo}
\NormalTok{mod1  <-}\StringTok{ }\KeywordTok{lm}\NormalTok{(y }\OperatorTok{~}\StringTok{ }\NormalTok{A }\OperatorTok{+}\StringTok{ }\NormalTok{D }\OperatorTok{+}\StringTok{ }\NormalTok{A}\OperatorTok{*}\NormalTok{D }\OperatorTok{+}\StringTok{ }\NormalTok{B}\OperatorTok{*}\NormalTok{C }\OperatorTok{+}\StringTok{ }\NormalTok{A}\OperatorTok{*}\NormalTok{C }\OperatorTok{+}\StringTok{ }\NormalTok{A}\OperatorTok{*}\NormalTok{B}\OperatorTok{*}\NormalTok{C }\OperatorTok{+}\StringTok{ }\NormalTok{B}\OperatorTok{*}\NormalTok{C}\OperatorTok{*}\NormalTok{D )}
\end{Highlighting}
\end{Shaded}

\begin{Shaded}
\begin{Highlighting}[]
\CommentTok{# Realizamos nuestro analisis de varianza}
\NormalTok{anva1 <-}\StringTok{ }\KeywordTok{aov}\NormalTok{(mod1)}
\KeywordTok{summary}\NormalTok{(anva1)}
\end{Highlighting}
\end{Shaded}

\begin{verbatim}
##             Df Sum Sq Mean Sq F value   Pr(>F)    
## A            1  38318   38318  19.510 0.021540 *  
## D            1 384090  384090 195.559 0.000792 ***
## B            1     18      18   0.009 0.929648    
## C            1    495     495   0.252 0.650161    
## A:D          1  89850   89850  45.747 0.006603 ** 
## B:C          1   6440    6440   3.279 0.167858    
## A:C          1   1785    1785   0.909 0.410748    
## A:B          1     68      68   0.035 0.864199    
## D:B          1     39      39   0.020 0.896786    
## D:C          1     11      11   0.005 0.946156    
## A:B:C        1    564     564   0.287 0.629211    
## D:B:C        1   1871    1871   0.952 0.401085    
## Residuals    3   5892    1964                     
## ---
## Signif. codes:  0 '***' 0.001 '**' 0.01 '*' 0.05 '.' 0.1 ' ' 1
\end{verbatim}

\[ y = \beta_0 + \beta_1x_1 +\beta_2x_2 + \beta_3x_3 + \beta_4x_4 + \beta_{14}x_1x_4 + \beta_{23}x_2x_3 + \beta_{13}x_1x_3 + \beta_{12}x_1x_2 + \beta_{42}x_4x_2 +\beta_{43}x_4x_3 + \beta_{123}x_1x_2x_3 + \beta_{423}x_4x_2x_3  \]
Donde:

\(x_i\) = 1, si el nivel del factor de i se encuentra alto

\(x_i\) = 0, si el nivel del factor de i se encuentra bajo

i = 1 , representa el factor de A

i = 2 , representa el factor de B

i = 3 , representa el factor de C

i = 4 , representa el factor de D

De los cuatro factores son de interés: el entrehierro ánodo-cátodo (A),
la presión en la cámara del reactor (B), el flujo del gas C2F6 (C) y la
potencia aplicada al cátodo (D) se ha llegado a las siguientes
conclusiones

a).- Los efectos que tiene influenciencia altamente muy significativa
sobre la rapidez de grabado para el nitruro de silicio son:

\begin{itemize}
\tightlist
\item
  D = La potencia aplicada al catodo
\end{itemize}

b).- Los efectos que tiene influenciencia muy significativa sobre la
rapidez de grabado para el nitruro de silicio son:

\begin{itemize}
\tightlist
\item
  entre A = El entrehierro ánodo-cátodo con D = La potencia aplicada al
  catodo
\end{itemize}

c).- Los efectos que tiene influenciencia significativa sobre la rapidez
de grabado para el nitruro de silicio son:

\begin{itemize}
\tightlist
\item
  A = El entrehierro ánodo-cátodo
\end{itemize}

d).- No se encontró diferencias significativas

\begin{itemize}
\item
  B = la presión en la cámara del reactor
\item
  C = el flujo del gas C2F6
\item
  entre B = la presión en la cámara del reactor con C = el flujo del gas
  C2F6
\item
  entre A = El entrehierro ánodo-cátodo con C = el flujo del gas C2F6
\item
  entre A = El entrehierro ánodo-cátodo con B = la presión en la cámara
  del reactor
\item
  entre D = La potencia aplicada al catodo con C = el flujo del gas C2F6
\item
  entre A = El entrehierro ánodo-cátodo , B = la presión en la cámara
  del reactor y C = el flujo del gas C2F6
\item
  entre D = La potencia aplicada al catodo , B = la presión en la cámara
  del reactor y C = el flujo del gas C2F6
\end{itemize}

\begin{Shaded}
\begin{Highlighting}[]
\CommentTok{# Obtenemos los parametros}
\KeywordTok{coefficients}\NormalTok{(mod1)}
\end{Highlighting}
\end{Shaded}

\begin{verbatim}
## (Intercept)           A           D           B           C         A:D 
##    777.9375    -48.9375    154.9375      1.0625      5.5625    -74.9375 
##         B:C         A:C         A:B         D:B         D:C       A:B:C 
##    -20.0625    -10.5625     -2.0625      1.5625      0.8125     -5.9375 
##       D:B:C 
##    -10.8125
\end{verbatim}

Del analisis de varianza se detecto que los efectos mas importantes son
de A, D y A:D

\[ y =  -48.9375x_1 + 154.9375x_4 + -74.9375x_1x_4 \]

\begin{Shaded}
\begin{Highlighting}[]
\KeywordTok{par}\NormalTok{(}\DataTypeTok{mfrow=}\KeywordTok{c}\NormalTok{(}\DecValTok{2}\NormalTok{,}\DecValTok{2}\NormalTok{))}
\KeywordTok{plot}\NormalTok{(mod1)}
\end{Highlighting}
\end{Shaded}

\begin{verbatim}
## hat values (leverages) are all = 0.8125
##  and there are no factor predictors; no plot no. 5
\end{verbatim}

\includegraphics{efade2019i_files/figure-latex/unnamed-chunk-8-1.pdf}

\begin{Shaded}
\begin{Highlighting}[]
\NormalTok{ri1<-}\KeywordTok{rstandard}\NormalTok{(mod1)}
\KeywordTok{shapiro.test}\NormalTok{(ri1)}
\end{Highlighting}
\end{Shaded}

\begin{verbatim}
## 
##  Shapiro-Wilk normality test
## 
## data:  ri1
## W = 0.87299, p-value = 0.03022
\end{verbatim}

\begin{Shaded}
\begin{Highlighting}[]
\KeywordTok{library}\NormalTok{(car)}
\end{Highlighting}
\end{Shaded}

\begin{verbatim}
## Loading required package: carData
\end{verbatim}

\begin{Shaded}
\begin{Highlighting}[]
\KeywordTok{ncvTest}\NormalTok{(mod1)}
\end{Highlighting}
\end{Shaded}

\begin{verbatim}
## Non-constant Variance Score Test 
## Variance formula: ~ fitted.values 
## Chisquare = 0.005323703, Df = 1, p = 0.94183
\end{verbatim}

\#\#Analisis Factorial 2x4

\hypertarget{un-telefono-inteligente-es-un-telefono-movil-que-ofrece-una-capacidad-de-computo-y-conectividad-mas-avanzadas-que-un-telefono-con-funciones-basica-contemporaneo.-los-datos-siguientes-son-las-calificaciones-de-seis-telefonos-inteligentes-de-cada-uno-de-los-cuatro-proveedores-tres-de-los-cuales-cuestan-150-o-mas-otros-tres-cuestan-menos-de-150.-las-calificaciones-tienen-un-valor-maximo-de-100-y-un-minimo-de-0.}{%
\paragraph{2.)- Un teléfono inteligente es un teléfono móvil que ofrece
una capacidad de cómputo y conectividad más avanzadas que un ``teléfono
con funciones'' básica contemporáneo. Los datos siguientes son las
calificaciones de seis teléfonos inteligentes de cada uno de los cuatro
proveedores, tres de los cuales cuestan \$ 150 o más, otros tres cuestan
menos de \$ 150. Las calificaciones tienen un valor máximo de 100 y un
mínimo de
0.}\label{un-telefono-inteligente-es-un-telefono-movil-que-ofrece-una-capacidad-de-computo-y-conectividad-mas-avanzadas-que-un-telefono-con-funciones-basica-contemporaneo.-los-datos-siguientes-son-las-calificaciones-de-seis-telefonos-inteligentes-de-cada-uno-de-los-cuatro-proveedores-tres-de-los-cuales-cuestan-150-o-mas-otros-tres-cuestan-menos-de-150.-las-calificaciones-tienen-un-valor-maximo-de-100-y-un-minimo-de-0.}}

\begin{verbatim}
Proveedor           
"AT&T((b1)" "Sprint
\end{verbatim}

(b2)" ``T-Mobile (b3)'' ``Verizon (b4)'' Costo  \$150 76 74 72 75 (a1)
74 69 71 73 69 68 71 73 Costo \textless{} \$150 69 69 71 72 (a2) 67 64
71 71 64 60 70 70

Datos en archivo teléfono.txt \#\#\#\# a).- Realice el Análisis de
Variancia, Pruebe la hipótesis correspondientes y de las conclusiones y
recomendaciones correspondiente. (4 puntos)

\begin{Shaded}
\begin{Highlighting}[]
\NormalTok{telefono<-}\KeywordTok{read.table}\NormalTok{(}\StringTok{"telefono.txt"}\NormalTok{,T)}
\KeywordTok{str}\NormalTok{(telefono)}
\end{Highlighting}
\end{Shaded}

\begin{verbatim}
## 'data.frame':    24 obs. of  3 variables:
##  $ calificaciones: int  76 74 69 69 67 64 74 69 68 69 ...
##  $ costo         : Factor w/ 2 levels "a1","a2": 1 1 1 2 2 2 1 1 1 2 ...
##  $ proveedor     : Factor w/ 4 levels "b1","b2","b3",..: 1 1 1 1 1 1 2 2 2 2 ...
\end{verbatim}

\begin{Shaded}
\begin{Highlighting}[]
\NormalTok{Calificación<}\OperatorTok{-}\NormalTok{telefono[,}\DecValTok{1}\NormalTok{]}
\NormalTok{Costo<-}\KeywordTok{as.factor}\NormalTok{(telefono[,}\DecValTok{2}\NormalTok{])}
\NormalTok{Proveedor<-}\KeywordTok{as.factor}\NormalTok{(telefono[,}\DecValTok{3}\NormalTok{])}
\NormalTok{mod<-}\KeywordTok{lm}\NormalTok{(Calificación~Costo}\OperatorTok{*}\NormalTok{Proveedor)}
\NormalTok{anva<-}\KeywordTok{aov}\NormalTok{(mod)}
\KeywordTok{summary}\NormalTok{(anva)}
\end{Highlighting}
\end{Shaded}

\begin{verbatim}
##                 Df Sum Sq Mean Sq F value  Pr(>F)   
## Costo            1  92.04   92.04  13.893 0.00183 **
## Proveedor        3  81.12   27.04   4.082 0.02490 * 
## Costo:Proveedor  3  33.46   11.15   1.683 0.21053   
## Residuals       16 106.00    6.63                   
## ---
## Signif. codes:  0 '***' 0.001 '**' 0.01 '*' 0.05 '.' 0.1 ' ' 1
\end{verbatim}

\hypertarget{hipotesis-del-efecto-principal-de-costo}{%
\subsection{Hipotesis del efecto principal de
Costo}\label{hipotesis-del-efecto-principal-de-costo}}

\(H_0\) : \(\alpha_i = 0\) , para i = 1,2,3

\(H_1\) : al menos dos \(\alpha_i != 0\) , para i = 1,2,3

Se rechaza la \(H_0\), a un nivel de significacion del 1\%, se ha
encontrado suficientes evidencia estadistica para rechazar la \(H_0\) de
que las categorias de los Costos tenga un efecto significativo sobre la
Calificación del teléfono.Por lo tanto, se puede aceptar de que existe
diferencias significativas entre al menos dos categorias de costos de
teléfono tienen un efecto significativo sobre su calificación

\hypertarget{hipotesis-del-efecto-principal-de-proveedor}{%
\subsection{Hipotesis del efecto principal de
Proveedor}\label{hipotesis-del-efecto-principal-de-proveedor}}

\(H_0\) : \(\beta_j\) = 0 , para j = 1,2,3

\(H_1\) : al menos dos \(\beta_j != 0\) , para j = 1,2,3

Se rechaza la \(H_0\), a un nivel de significacion del 5\%, se ha
encontrado suficientes evidencia estadistica para rechazar la \(H_0\) de
que las proveedores tenga un efecto significativo sobre la calificación
del teléfono.Por lo tanto, se puede aceptar de que existe diferencias
significativas entre al menos dos proveedores tienen un efecto
significativo sobre la calificación del teléfono.

\hypertarget{hipotesis-del-efecto-de-interaccion-entre-el-costo-y-el-proveedor}{%
\subsection{Hipotesis del efecto de interacción entre el Costo y el
Proveedor}\label{hipotesis-del-efecto-de-interaccion-entre-el-costo-y-el-proveedor}}

\(H_0\) : \((\alpha * \beta )_ = 0\) , para i,j = 1,2,3

\(H_1\) : al menos dos \((\alpha * \beta ) != 0\) , para i,j = 1,2,3

Se acepta \(H_0\), a un nivel de significacion del 10\%, no se ha
encontrado suficiente evidencia estadistica para rechazar la \(H_0\) ,es
decir que no existe ning'un efecto de la interaci'on entre la categoria
del costo y el proveedor influyan sobre la calificación del teléfono.

\hypertarget{b.--teniendo-en-cuenta-los-resultados-obtenido-en-a-realice-la-prueba-de-tukey-correspondiente}{%
\paragraph{b).- Teniendo en cuenta los resultados obtenido en (a),
realice la prueba de Tukey
correspondiente}\label{b.--teniendo-en-cuenta-los-resultados-obtenido-en-a-realice-la-prueba-de-tukey-correspondiente}}

(6 puntos)

De la parte a, se ha visto que la categoria de costo y los proveedores
tiene efecto sobre la calificación del teléfono, se procedera a realizar
la prueba de Tukey para la Cada uno de ellos

\begin{Shaded}
\begin{Highlighting}[]
\KeywordTok{library}\NormalTok{(multcomp)}
\end{Highlighting}
\end{Shaded}

\begin{Shaded}
\begin{Highlighting}[]
\NormalTok{tHSD <-}\StringTok{ }\KeywordTok{TukeyHSD}\NormalTok{(anva)}
\KeywordTok{summary}\NormalTok{(tHSD)}
\end{Highlighting}
\end{Shaded}

\begin{verbatim}
##                 Length Class  Mode   
## Costo             4    -none- numeric
## Proveedor        24    -none- numeric
## Costo:Proveedor 112    -none- numeric
\end{verbatim}

\begin{Shaded}
\begin{Highlighting}[]
\NormalTok{tHSD}
\end{Highlighting}
\end{Shaded}

\begin{verbatim}
##   Tukey multiple comparisons of means
##     95% family-wise confidence level
## 
## Fit: aov(formula = mod)
## 
## $Costo
##            diff       lwr       upr     p adj
## a2-a1 -3.916667 -6.144249 -1.689084 0.0018334
## 
## $Proveedor
##            diff        lwr      upr     p adj
## b2-b1 -2.500000 -6.7516077 1.751608 0.3643388
## b3-b1  1.166667 -3.0849410 5.418274 0.8601187
## b4-b1  2.500000 -1.7516077 6.751608 0.3643388
## b3-b2  3.666667 -0.5849410 7.918274 0.1039730
## b4-b2  5.000000  0.7483923 9.251608 0.0185696
## b4-b3  1.333333 -2.9182743 5.584941 0.8063964
## 
## $`Costo:Proveedor`
##                   diff         lwr        upr     p adj
## a2:b1-a1:b1 -6.3333333 -13.6093432  0.9426765 0.1129895
## a1:b2-a1:b1 -2.6666667  -9.9426765  4.6093432 0.8976115
## a2:b2-a1:b1 -8.6666667 -15.9426765 -1.3906568 0.0140595
## a1:b3-a1:b1 -1.6666667  -8.9426765  5.6093432 0.9911124
## a2:b3-a1:b1 -2.3333333  -9.6093432  4.9426765 0.9451981
## a1:b4-a1:b1  0.6666667  -6.6093432  7.9426765 0.9999756
## a2:b4-a1:b1 -2.0000000  -9.2760099  5.2760099 0.9752592
## a1:b2-a2:b1  3.6666667  -3.6093432 10.9426765 0.6614247
## a2:b2-a2:b1 -2.3333333  -9.6093432  4.9426765 0.9451981
## a1:b3-a2:b1  4.6666667  -2.6093432 11.9426765 0.3896498
## a2:b3-a2:b1  4.0000000  -3.2760099 11.2760099 0.5673822
## a1:b4-a2:b1  7.0000000  -0.2760099 14.2760099 0.0638225
## a2:b4-a2:b1  4.3333333  -2.9426765 11.6093432 0.4752993
## a2:b2-a1:b2 -6.0000000 -13.2760099  1.2760099 0.1484555
## a1:b3-a1:b2  1.0000000  -6.2760099  8.2760099 0.9996326
## a2:b3-a1:b2  0.3333333  -6.9426765  7.6093432 0.9999998
## a1:b4-a1:b2  3.3333333  -3.9426765 10.6093432 0.7517157
## a2:b4-a1:b2  0.6666667  -6.6093432  7.9426765 0.9999756
## a1:b3-a2:b2  7.0000000  -0.2760099 14.2760099 0.0638225
## a2:b3-a2:b2  6.3333333  -0.9426765 13.6093432 0.1129895
## a1:b4-a2:b2  9.3333333   2.0573235 16.6093432 0.0075749
## a2:b4-a2:b2  6.6666667  -0.6093432 13.9426765 0.0852319
## a2:b3-a1:b3 -0.6666667  -7.9426765  6.6093432 0.9999756
## a1:b4-a1:b3  2.3333333  -4.9426765  9.6093432 0.9451981
## a2:b4-a1:b3 -0.3333333  -7.6093432  6.9426765 0.9999998
## a1:b4-a2:b3  3.0000000  -4.2760099 10.2760099 0.8321663
## a2:b4-a2:b3  0.3333333  -6.9426765  7.6093432 0.9999998
## a2:b4-a1:b4 -2.6666667  -9.9426765  4.6093432 0.8976115
\end{verbatim}

\$Costo diff lwr upr p adj a2-a1 -3.916667 -6.144249 -1.689084 0.0018334

A un nivel de significación del 0.1\%, se a encontrados suficiente
evidencia para afirmar que la media de la calificación de un teléfono
que cuesta menos de 150 dolares es diferente a la media de la
calificación de un teléfono que cuesta mas de 150 dolares.

\$Proveedor diff lwr upr p adj

b4-b2 5.000000 0.7483923 9.251608 0.0185696

A un nivel de significación del 1\%, se a encontrados suficiente
evidencia para afirmar que la media de la calificaciòn de un teléfono
del proveedor Verizon es diferente a la media de la calificación de un
teléfono del proveedor Sprint

Para los otros casos no existe diferencias significativa


\end{document}
